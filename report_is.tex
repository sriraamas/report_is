\documentclass[12pt,titlepage,a4paper]{article}
\begin{document}
\title{Independent Study Report - Spring 2014}
\author{Sriraam Appusamy Subramanian}
\date{May 2014}
\maketitle

\begin{abstract}
Operating system allows processes in the system to share the computing, storage and peripheral resources with safety and isolation. Much of the mechanisms involved in achieving the above capabilities trades off performance/latency for flexibility which remains acceptable for most of the day-to-day applications. However,for certain applications like memcached, SDN controller,etc., whose performance is tied to their disk/peripheral device access capabilities, this flexibility is less desirable than performance. The goal of this independent study was to evaluate and explore existing systems that allow for direct peripheral device assignment to applications, exclusive or not.Intel's DataPlane Development Kit (DPDK) is a software library that uses modified userspace drivers, polling, prefetching/pre-allocation of buffers/queues and other enhancements to allow upto 80Mpps packet processing capabilities on Intel Architecture platforms. However DPDK supports restrictive number of NIC types.  Intel VT-d is an hardware virtualization technique that provides dedicated I/O Memory management unit and Interrupt remapping capabilities that could allow for exclusive device usage by an application. SUD is an existing system, developed using Usermode Linux(UML) that aims to test malicious device drivers by running them in userspace. Since SUD was implemented in linux kernel 2.6.x, it became necessary to port the project to linux kernel 3.10.x to allow for further testing and benchmarking. Although most of the porting is done, testing still remains incomplete due to failed porting of broadcom driver BNX2 into UMLinux. 
\end{abstract}

\section{DPDK}
Intel's DataPlane Development Kit aims to improve packet processing performance in Intel platforms. This is achieved by providing applications with customized software library that significantly improves peripheral device access performance. This library provides a simple API interface for buffer management, queue management, poll-mode capable userspace drivers and packet-flow classification. Further several abstraction layers are added to both userspace and kernel-space to help communicate with the peripheral device in traditional Linux-based operating systems. An experiment was setup in Emulab with gpu2 node and 2 Network Interace Cards(NICs) were assigned to DPDK. Following this, a sample application was run to verify DPDK. After a quick analysis of the existing openflow controllers and their performance metrics, NOX was chosen to be modified to support DPDK. However,limited types of hardware supported by DPDK warranted continued exploration for other mechanisms.
\section{SUD}

\section{Porting to 3.x kernel}
Some lines regarding Porting 
\section{Results}

\section{Conclusion}
\end{document}
